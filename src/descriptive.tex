\chapter{Descriptive Statistics}

\section{What is Statistics?}

``Statistics`` is a science of decision making on the basis of sample observations 
drawn from a population under uncertainty. That is, it is a mathematical discipline 
concerned with the collection of data, summarization of data, analysis of data and
interpretation of data toward a valid decision.

Given a sample (a set of outcomes), we are to say (infer) about the population or 
the model. Statistics primarily deals with situations in which the occurrence of some event 
can't be predicted with certainty. Here are the major objectives of statistics:
\begin{itemize}
    \item Make inference about a population from an analysis of information 
contained in the sample data.
    \item To make assessments of the extent of uncertainty involved in these inferences.
    \item A third objective, no less important, is to design the process and the extent of sampling so 
that the observations from a basis for drawing valid and accurate inferences.  
\end{itemize}

\section{Collecting Data}

Statistical data are frequently obtained by a process in which the desired information is 
obtained from the source, either by having an enumerator visit to the informant, ask the 
necessary questions and enter the replies on a schedule, or by sending to the informant a 
list of questions or some questionnaire which he may answer at his 
convenience. The term ``questionnaire`` means a list of certain systematically arranged 
questions relating to the subject of enquiry. It is necessary that questionnaire is designed 
with due care so that necessary data may be easily collected. 

 In the schedule one finds a list of items, on which information will be collected, 
the exact forms of the questions to be put to the informants are not given and task of 
questioning, explaining the desired information is left to the investigator. 

\section{Presentation of Data}

\subsection{Bar chart}

A Bar diagram which consists of a number of rectangles (usually called bars) is used for 
one-dimensional comparison. It is used to show absolute changes in magnitudes overtime 
(chronological) or space (geographical/regional). Changes in time or space, as the case may 
be, are shown along the x-axis with equally spaced magnitudes. Rectangles of equal width 
are drawn with lengths varying with the magnitude represented. While a line graph is not 
suitable for representation of data classified geographically or qualitatively, a bar diagram 
is suitable for representation of such data. Vertical bars should also be used for data classified quantitatively. When making 
comparisons of data classified qualitatively or geographically, on the other hand, horizontal 
bars are generally used. 

\subsection*{Pie chart}

When an aggregate is divided into different components, we may be interested in the 
relative importance of the different components, rather than their absolute contribution to 
the aggregate. For representing breakdown of an aggregate into components a pie diagram 
is used. For pie diagram, one circle is used and the area enclosed by it being taken as 100. It 
is then divided into a number of sectors by drawing angles at the centre, the area of each 
sector representing the corresponding percentage. Since the full angle at the center is  
360°, it is clear that for any particular category the angle (in degrees) should be 3.6 times 
the corresponding percentage.

When observations on discrete or continuous variables are available on a single 
characteristic of a large number of members often it becomes necessary to condense the 
data as far as possible without losing any information of interest. If the data is 
non-frequency type, then the first step of condensation is to classify different values or is 
to divide the observed range of the variable into a suitable member of groups or classes, 
according to their increasing order in terms of magnitude and to record the number of 
observations corresponding to each distinct value or falling in each class. 

\begin{figure}[ht]
    \centering
    \makebox[\textwidth]{\includegraphics[scale=0.6]{./images/pie_chart.pdf}}
\end{figure}

\section{Frequency distribution}

When observations on discrete or continuous variables are available on a single 
characteristic of a large number of members often it becomes necessary to condense the 
data as far as possible without losing any information of interest. If the data is 
non-frequency type, then the first step of condensation is to classify different values or is 
to divide the observed range of the variable into a suitable member of groups or classes, 
according to their increasing order in terms of magnitude and to record the number of 
observations corresponding to each distinct value or falling in each class.  