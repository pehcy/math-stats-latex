\chapter{Nonparametric Statistics II -- Procedures that utilize data from three or more related samples}

Frequently, we can greatly improve the ability to detect group differences in the variable of
interest by dividing subjects into homogeneous subgroups, called blocks, and then making
comparisons among subjects within the subgroups. We can do this by using randomized complete
block design (two-way ANOVA). This technique extends the two-sample paired comparison model
discussed in Chapter 4 to the case in which several samples are available for analysis. Thus, for
three or more samples, a block is composed of three or more subjects, more generally referred to as
experimental units, who are more homogeneous with respect to each other then with respect to
subjects in another block. We could form blocks on the basis of age, education and physical
condition. In certain situations a single subject may be a block.

\section{Friedman Two-Way Analysis Of Variance by ranks}

This test is a nonparametric analogue of the parametric two-way analysis of variance. We
perform calculations on ranks, which may be derived from observations measured on a higher scale
or may be the original observations themselves. The procedure may be used when for one reason or
another it is undesirable to use the parametric two-way ANOVA. For example, the investigator may
be unwilling to assume that the sampled populations are normally distributed, a requirement for the
valid use of the parametric test. Also, in some cases only ranks may be available for analysis.

The objective is to determine if we may conclude from sample evidence that there is a
difference among treatment effects. We reason that if treatments do not differ in their effects, the
median response of a population of subjects receiving a given treatment will be the same as the
median response of a population of subjects receiving any one of the other treatments under study,
after the effect of the blocking variable has been removed. Thus, if we are comparing $k$ samples

\subsection{Ties}

Theoretically, no ties should occur, since the variable whose values are ranked is assumed to
be continuous. In practice, however, ties do occur, and we give tied observations the mean of the
rank positions for which they are tied. Note that only ties within a given block are of concern.

\begin{equation}
    W = \frac{12 \sum^k_{j=1} R^2_j - 3b^2 k(k+1)^2}{b^2k(k^2 - 1) - b \left( \sum t^3 - \sum t \right)}
\end{equation}

The Friedman test is based on $b$ sets of ranks, and the treatments are ranked separately in
each set. Such a ranking scheme allows for intrablock comparisons only, since interblock
comparisons are not meaningful. When the number of treatments is small, this may pose a
disadvantage. When situations arise in which comparability among blocks is desirable, the method
of aligned ranks may be employed.

\begin{enumerate}
    \item Subtract from each observation within a block some measure of location such as the block
mean or median. The resulting differences, called aligned observation, which keep their
identities with respect to the block and treatment combination to which they belong, are then
ranked from 1 to $\fbox{$kb$}$ relative to each other (the same as the \textbf{Kruskal-Wallis Test}).

    \item If there is no treatment effect, we would expect each of the blocks to receive approximately
the same sequence of aligned ranks. We would expect the treatment rank totals to be about
equal. In the absence of ties, the aligned-ranks test statistic for the randomized complete
block design is
\begin{equation}
    T = \frac{\displaystyle (k-1) \left[ \sum_{j=1}^{k} \widehat{R}^2_j - \frac{kb^2}{4}(kb+1)^2 \right]}{\displaystyle \left[ \frac{kb(kb+1)(2kb+1)}{6}\right] - \frac{1}{k} \sum_{i=1}^{b} \widehat{R}^2_j } \sim \chi^2_{k-1}.
\end{equation}

    \item If ties are present, replace the denominator of $T$ with 
        \begin{equation}
            \sum^b_{i=1} \sum^{k}_{j=1} \widehat{R}^2_{ij} - \frac{1}{k} \sum^b_{i=1} R^2_i.
        \end{equation}

\end{enumerate}

\section{Page's Test for Ordered Alternatives}

\begin{definition}[Page's test for Ordered Alternatives]
    The testing hypotheses are 
    \[
        H_0 : \tau_1 = \tau_2 = \ldots = \tau_k
    \]
    versus
    \[
        H_1: \text{The treatment effects } \tau_1, \tau_2, \ldots, \tau_k \text{ are ordered in the form of } \tau_1 \leq \tau_2 \leq \ldots \leq \tau_k.
    \]
    The test statistic is 
    \[
        L = \sum^k_{j = 1} jR_j = R_1 + 2R_2 + 3R_3 + \cdots + kR_k
    \]
    where $R_j$ for $j = 1, 2, 3, \ldots, k$ are treatment rank sums obtained in the manner
\end{definition}

\begin{theorem}[Large Sample Approximation for Page's Test]
    For large sample size, we use the test statistic:
    \begin{equation}
        z = \frac{\displaystyle L - \frac{bk(k+1)^2}{4} }{\sqrt{\displaystyle \frac{b(k^3 - k)^2}{144(k-1)}}} \sim N(0,1)
    \end{equation}
    Note that the large sample statistic follows standard normal instead of chi-square distribution.
\end{theorem}

\section{Durbin's Test for Incomplete Block Designs}

In designing an experiment, the investigator may find that it is impossible or impractical to
construct a randomized complete block design of the type discussed so far. It may be impossible or
impractical to apply all treatments to each block. This becomes an important problem when the
number of treatments is large and the size of the blocks is limited. For example, we are going to
compare the effects of seven treatments by administering the treatments to laboratory animals, with
litters serving as block. Because the subjects must meet certain criteria, we can use only three
animals from each litter. These conditions suggest that we use an incomplete block design, since we
cannot administer each treatment to an animal from each litter.

The particular type of incomplete block design with which we are concerned is the balanced
incomplete block design. In this design every possible pair of treatments appears the same number
of times. Further, the balanced incomplete block design requires that each block contain the same
number of subjects and that each treatment occur the same number of times.