\chapter{Sampling Distributions}

\section{Snedecor's $F$-distribution}

TheF-distribution was named in honor of Sir Ronald Fisher by George
Snedecor. F-distribution arises as the distribution of a ratio of variances.
Like, the other two distributions this distribution also tends to normal 
distribution as $\nu_1$ and $\nu_2$ become very large. The
following figure illustrates the
 shape of the graph of this distribution for various degrees of freedom.

\begin{theorem}
If the random variable $X$ is $F$-distributed with degrees 
of freedom $\nu_1$ and $\nu_2$, then its mean is 
\begin{equation}
    \mathbb{E}[X] = \begin{cases}
        \displaystyle \frac{\nu_2}{\nu_2 - 2} & \text{if } \nu_2 \geq 3\\
        DNE & \text{if } \nu_2 = 1,2
    \end{cases}
\end{equation}
and the variance is 
\begin{equation}
    Var[X] = \begin{cases}
        \displaystyle \frac{2\nu^2_2 (\nu_1 + \nu_2 - 2)}{\nu_1 (\nu_2 - 2)^2 (\nu_2 - 4)} & \text{if } \nu_2 \geq 5\\
        DNE & \text{if } \nu_2 = 1,2,3,4.
    \end{cases}
\end{equation}
\end{theorem}

\begin{theorem}
    If a random variable $X \sim F(\nu_1, \nu_2)$, then its reciprocal $\mfrac{1}{X} \sim F(\nu_1, \nu_2)$.
\end{theorem}

\begin{theorem}
    If the random variables $U \sim \chi^2(\nu_1)$ and $V \sim \chi^2(\nu_2)$, and 
    $U$ and $V$ are independent, then 
    \begin{equation}
        \frac{U/\nu_1}{V/\nu_2} \sim F(\nu_1, \nu_2).
    \end{equation}
\end{theorem}

\begin{example}
    Let $X_1, X_2, \ldots, X_4$ and $Y_1, Y_2, \ldots, Y_5$ be two random samples 
    of size $4$ and $5$, respectively, from a standard normal population. What is 
    the variance of the statistic 
    \[
        T = \left( \frac{5}{4} \right) \frac{X^2_1 + X^2_2 + X^2_3 + X^2_4}{Y^2_1 + Y^2_2 + Y^2_3 + Y^2_4 + Y^2_5}.
    \]
\end{example}
\begin{solution}
    Since the population is standard normal, we have 
    \[
        X^2_1 + X^2_2 + X^2_3 + X^2_4 \sim \chi^2(4).
    \]
    Similarly, 
    \[
        Y^2_1 + Y^2_2 + Y^2_3 + Y^2_4 + Y^2_5 \sim \chi^2(5).
    \]
    Therefore, 
    \begin{align*}
        T &= \left( \frac{5}{4} \right) \frac{X^2_1 + X^2_2 + X^2_3 + X^2_4}{Y^2_1 + Y^2_2 + Y^2_3 + Y^2_4 + Y^2_5}\\
        &= \frac{\displaystyle \frac{X^2_1 + X^2_2 + X^2_3 + X^2_4}{4} }{ \displaystyle \frac{Y^2_1 + Y^2_2 + Y^2_3 + Y^2_4 + Y^2_5}{5}}\\
        &\sim F(4,5).
    \end{align*}

    Applying theorem, the variance of this statistic is 
    \begin{align*}
        Var[T] &= Var[F(4,5)]\\
        &= \frac{2 (5)^2 (4 + 5 - 2)}{4 (5 - 2)^2 (5 - 4)}\\
        &= \frac{350}{36}.
    \end{align*}
\end{solution}