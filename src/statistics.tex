\chapter{Review on Statistics}

\section{Distinguishing Between a Population and a Sample}

Think of a \textbf{population} like all the cookies in a giant cookie jar - it's the entire group we're interested in studying. A \textbf{sample} is like taking a handful of cookies from that jar to taste and learn about all the cookies without eating them all.

\textbf{Population}: The complete set of all individuals, objects, or measurements that we want to study or make conclusions about.

\textbf{Sample}: A subset or portion of the population that we actually observe, measure, or collect data from.
When the sample subset contained every member of the population, it is called a \textbf{census}.

\begin{example}
Suppose you want to know the average height of all students at your university (that's about 25,000 students). 
\begin{itemize}
    \item[{\color{blue!55!white} \ding{104}}] The \textbf{population} would be all 25,000 students at the university.
    \item[{\color{blue!55!white} \ding{104}}] The \textbf{sample} might be 500 students that you randomly select and actually measure.
\end{itemize}
You use the average height from your sample of 500 students to estimate the average height of the entire population of 25,000 students. This saves time and money compared to measuring every single student!
\end{example}


\section{Sampling Techniques}

\begin{table}[h!]
\renewcommand{\arraystretch}{1.3}
\begin{tabularx}{\textwidth}{|p{2.5cm}|X|p{3cm}|p{3cm}|}
\hline
\textbf{\color{fireEngRed}Sampling Technique} & \textbf{\color{fireEngRed}Example} & \textbf{\color{fireEngRed}Advantages} & \textbf{\color{fireEngRed}Limitations} \\
\hline
Simple random sampling & The names of all 1,000 children are placed into a computer database. The computer is then instructed to randomly select 100 names. These children and their parents are then contacted. & Representative of the population & May be difficult to obtain the list; May be more expensive \\
\hline
Stratified random sampling & The names of all 1,000 children are placed into a computer database and organized by grade (sixth, seventh, eighth). The computer is then instructed to randomly select 35 names from each of the three grades. These children and their parents are then contacted. & Representative of the population & May be difficult to obtain the list; May be more expensive \\
\hline
Convenience sampling & The researcher knows one of the middle-school teachers, and the teacher volunteers her 35 students for the study. These children and their parents are then contacted. & Simple; Easy; Convenient; No complete member list needed & May not be representative of the population \\
\hline
Quota sampling & Using the middle-school directory, the researcher selects the first 20 sixth-grade boys, the first 20 sixth-grade girls, the first 20 seventh-grade boys, the first 20 seventh-grade girls, the first 20 eighth-grade boys, and the first 20 eighth-grade girls. These children and their parents are then contacted. & Simple; Easy; Convenient; No complete member list needed & May not be representative of the population \\
\hline
\end{tabularx}
\end{table}

\subsection{Simple Random Sampling}

The probability sampling can be done in specifying the probability 
of a member being chosen into the sample. You can think of this as a weight 
assigned to each member of the population. 

One caution before we proceed to other sampling techniques: Random sampling is 
totally different from random assignment. Random sampling is used to select a sample from a population, 
while random assignment is a random process of locating participants into different experimental 
groups or labels.

\subsection{Stratified Random Sampling}

To stratify means to classify or separate the individuals in a population into 
different groups or strata based on some common characteristic.